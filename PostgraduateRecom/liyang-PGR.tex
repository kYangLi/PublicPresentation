%! TEX program = xelatex

\documentclass{beamer}

  \usepackage[utf8]{inputenc}
  \usepackage{utopia} %font utopia imported
  \usepackage{amsmath}
  \usepackage{xeCJK}
  
  \usetheme{Madrid}
  \usecolortheme{default}
  
 %v-v-v-v-v-v-v-v-v-v-v-v-v-v-v-v-v-v-v-v-v-v-v-v-v-v-v-v-v-v-v-v-
  %This block of code defines the information to appear in the
  %Title page
  \title[Postgraduate Recommendation] %optional
  {保研经验交流}
  
  \subtitle{(一个掺杂了各种主观认识的报告)}
  
  \author[Yang Li]
  {李洋\inst{1}}
  
  \institute[JLU] 
  {
    \inst{1}%
    Department of Physics\\
    Jilin University 
  }
  
  \date[JLU Physics 2018]
  {May 2018}
  
  %\logo{\includegraphics[height=1.5cm]{jlu.png}
  %End of title page configuration block
  %^-^-^-^-^-^-^-^-^-^-^-^-^-^-^-^-^-^-^-^-^-^-^-^-^-^-^-^-^-^-^-^-
  
  %v-v-v-v-v-v-v-v-v-v-v-v-v-v-v-v-v-v-v-v-v-v-v-v-v-v-v-v-v-v-v-v-
  %The next block of commands puts the table of contents at the 
  %beginning of each section and highlights the current section:
  
  \AtBeginSection[]
  {
    \begin{frame}
      \frametitle{Table of Contents}
      \tableofcontents[currentsection]
    \end{frame} 
  }
  %^-^-^-^-^-^-^-^-^-^-^-^-^-^-^-^-^-^-^-^-^-^-^-^-^-^-^-^-^-^-^-^-
  
  \begin{document}
  
    %v-v-v-v-v-v-v-v-v-v-v-v-v-v-v-v-v-v-v-v-v-v-v-v-v-v-v-v-v-v-v-v-
    %The next statement creates the title page.
    \frame{\titlepage}
    %^-^-^-^-^-^-^-^-^-^-^-^-^-^-^-^-^-^-^-^-^-^-^-^-^-^-^-^-^-^-^-^-
    
    %v-v-v-v-v-v-v-v-v-v-v-v-v-v-v-v-v-v-v-v-v-v-v-v-v-v-v-v-v-v-v-v-
    %This block of code is for the table of contents after
    %the title page
    \begin{frame}
    \frametitle{Table of Contents}
    \tableofcontents
    \end{frame}
    %^-^-^-^-^-^-^-^-^-^-^-^-^-^-^-^-^-^-^-^-^-^-^-^-^-^-^-^-^-^-^-^-
    
    \section{一些前置声明}  
    
    %v-v-v-v-v-v-v-v-v-v-v-v-v-v-v-v-v-v-v-v-v-v-v-v-v-v-v-v-v-v-v-v-
    %Changing visivility of the text
    \begin{frame}
    \frametitle{一些前置声明}
      首先声明本报告的特性:

      \begin{description}
        \small
        \item[不可重复] 由于不同人情况不同, 本报告内容可能并不具备可重复性, 实际操作时请勿完全以此为准.
        \item[缺乏细节] 由于年代久远, 对一部分细节操作的记忆已相对模糊, 因此目前很难事无巨细地阐明这一过程.
        \item[存在引用] 本报告中的部分方法引用自13级学长, 遇此情况会特别说明. 
        \item[内容开源] 本报告内容完全开源, 您可以以任何非盈利目的分享此中信息. 通过访问作者GitHub相关项目网站, 获取此PPT源码:\url{}
      \end{description}
      \begin{block}{Remark}
        本
      \end{block}
    \end{frame}
    %^-^-^-^-^-^-^-^-^-^-^-^-^-^-^-^-^-^-^-^-^-^-^-^-^-^-^-^-^-^-^-^-

    \section{呈递申请材料}  
    
    %v-v-v-v-v-v-v-v-v-v-v-v-v-v-v-v-v-v-v-v-v-v-v-v-v-v-v-v-v-v-v-v-
    %Changing visivility of the text
    \begin{frame}
    \frametitle{Brief Research Status Overview}
    Cite Data Analysis Result I:
    你好!!
    \end{frame}
    %^-^-^-^-^-^-^-^-^-^-^-^-^-^-^-^-^-^-^-^-^-^-^-^-^-^-^-^-^-^-^-^-

    \section{参加(笔试与)面试}  
    
    %v-v-v-v-v-v-v-v-v-v-v-v-v-v-v-v-v-v-v-v-v-v-v-v-v-v-v-v-v-v-v-v-
    %Changing visivility of the text
    \begin{frame}
    \frametitle{Brief Research Status Overview}
    Cite Data Analysis Result I:
    你好!!
    \end{frame}
    %^-^-^-^-^-^-^-^-^-^-^-^-^-^-^-^-^-^-^-^-^-^-^-^-^-^-^-^-^-^-^-^-

    \section{确认预录取信息}  
    
    %v-v-v-v-v-v-v-v-v-v-v-v-v-v-v-v-v-v-v-v-v-v-v-v-v-v-v-v-v-v-v-v-
    %Changing visivility of the text
    \begin{frame}
    \frametitle{Brief Research Status Overview}
    Cite Data Analysis Result I:
    你好!!
    \end{frame}
    %^-^-^-^-^-^-^-^-^-^-^-^-^-^-^-^-^-^-^-^-^-^-^-^-^-^-^-^-^-^-^-^-

    \section{总结}  

    %v-v-v-v-v-v-v-v-v-v-v-v-v-v-v-v-v-v-v-v-v-v-v-v-v-v-v-v-v-v-v-v-
    %Changing visivility of the text
    \begin{frame}
      \frametitle{Brief Research Status Overview}
      Cite Data Analysis Result I:
      你好!!
      \end{frame}
      %^-^-^-^-^-^-^-^-^-^-^-^-^-^-^-^-^-^-^-^-^-^-^-^-^-^-^-^-^-^-^-^-
  \end{document}