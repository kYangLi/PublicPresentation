\documentclass{beamer}

%%% Usepackage
% Basic
\usepackage{amsmath}   % Math input support
\usepackage{amssymb}   % Math input support
\usepackage{booktabs}  % 3 line table show
\usepackage{float}     % Figure and Table env.
\usepackage{makecell}  % Mult. lines in one table cell
\usepackage{bm}
\usepackage{bbm}
\usepackage{hyperref}

% Color Support
\usepackage{xcolor}
\definecolor{commcolor}{rgb}{0,0.6,0}
\definecolor{rulesepcolor}{rgb}{0.2,0.2,0.2}
\definecolor{stringcolor}{rgb}{0.58,0,0.82}
\definecolor{backcolor}{rgb}{0.93,0.87,0.89}
\definecolor{backcolor2}{rgb}{1,1,0.9}
\definecolor{pergray}{rgb}{0.88,0.88,0.88}

% Coding block
\usepackage{listings}
\lstset{
  language=python,
  basicstyle=\tiny\ttfamily,
  captionpos=b,
  backgroundcolor=\color{backcolor2},
  commentstyle=\color{commcolor},
  escapeinside={\%*}{*)},
  keywordstyle=\color{blue},
  stringstyle=\color{stringcolor}\ttfamily,
  frame=none,
  rulesepcolor=\color{rulesepcolor},
  numbers=left,
  numbersep=4pt,
  numberstyle={\color[RGB]{170,170,170}},
}

% Coding block smooth
\usepackage{tikz}
\usepackage[framemethod=tikz,skipbelow=\topskip,skipabove=\topskip]{mdframed}
\mdfsetup{
  leftmargin=0pt,
  rightmargin=0pt,
  backgroundcolor=backcolor2,
  middlelinecolor=backcolor2,
  roundcorner=5pt,
}
\usepackage{etoolbox}
\BeforeBeginEnvironment{lstlisting}{\begin{mdframed}\vspace{-1em}}
\AfterEndEnvironment{lstlisting}{\vspace{-1em}\end{mdframed}}

% Inline coding
\usepackage{newverbs}
\newverbcommand{\cverb}
  {\setbox\verbbox\hbox\bgroup}
  {\egroup\tcbox{\color{purple}\box\verbbox}}

% Picture shadow box & Inline coding bg
\usepackage[many]{tcolorbox}
\tcbset{
  on line,
  boxsep=1pt, 
  left=1pt, right=1pt, top=1pt, bottom=1pt,
  colframe=backcolor, 
  colback=pergray,  
  highlight math style={enhanced}
}

%%% New command
\newcommand{\purple}{\textcolor{purple}}
\newcommand{\gray}{\textcolor{gray}}
%%% Page style
\setbeamertemplate{navigation symbols}{}
\usefonttheme{professionalfonts}
\usetheme{Madrid}
\usecolortheme{default}

%%% Pages
%+===================================================================+%
\title[DMI]{Dzyaloshinskii-Moriya interaction}

\author[Yang Li]{
  Yang Li\inst{1}}  
\institute[CMT Tsinghua Univ.]{
  \inst{1} Department of Physics\\Tsinghua University 
}

\date[Jul. 2022]{Jul. 2022}
%-===================================================================-%

\begin{document}
  %+===================================================================+%
  \frame{\titlepage}
  %-===================================================================-%
  
  %+===================================================================+%
  \begin{frame}{Mechanisms of DMI \(\rhd\) Bases}
    \begin{block}{}
    We use the wannier function as the bases to expand the Hamiltionian. 
    \begin{equation}
      \left\{\omega_{n\uparrow}(\bm{r}-\bm{R}), \omega_{n\downarrow}(\bm{r}-\bm{R})\right\}
    \end{equation}
    And the \(\widehat{\alpha}_{n\uparrow}(\bm{R})\) and \(\widehat{\alpha}_{n\downarrow}^\dagger(\bm{R})\) are the annihilation and the creation operators of the electrons in the state \(\omega_{n\uparrow}(\bm{r}-\bm{R})\), etc.\\
    \end{block}

    Then the Hamiltionian of a system after considering spin-orbit coupling (SOC) can be writen as, 
  \end{frame}
  %-===================================================================-% 

  %+===================================================================+%
  \begin{frame}{Mechanisms of DMI \(\rhd\) Hamiltionian} 
    \begin{block}{}
      \begin{equation}
        \widehat{H}_1 = \dfrac{\widehat{\bm{p}}^2}{2m} + V(\widehat{\bm{r}}) + \dfrac{\hbar}{2m^2c^2}\widehat{\bm{S}}\cdot[\nabla{}V(\widehat{\bm{r}})\times\widehat{\bm{p}}]
      \end{equation}
      The last term is SOC term (\(\widehat{H}_{\text{SOC}}\)) drive from the Dirac equation\footnote{\tiny\href{http://alma.karlov.mff.cuni.cz/hamrle/teaching/lectures/hamrle_spin-orbit-coupling.pdf}{Spin-orbit coupling: Dirac equation}}. 
    \end{block}
  
    Suppose \(V(\bm{r}) = V(|\bm{r}|) = V(r)\),
    \begin{equation}\begin{aligned}
      H_{\text{SOC}} &= \dfrac{\hbar}{2m^2c^2}\widehat{\bm{S}}\cdot[\dfrac{\mathrm{d}V(r)}{\mathrm{d}r}\dfrac{\widehat{\bm{r}}}{r}\times\widehat{\bm{p}}]\\
      &= \lambda\widehat{\bm{L}}\cdot\widehat{\bm{S}}\\
      &= \lambda\left(\widehat{L}_z\widehat{S}_z + \dfrac{1}{2}(\widehat{L}_+\widehat{S}_-+\widehat{L}_-\widehat{S}_+)\right)
    \end{aligned}\end{equation}


  \end{frame}
  %-===================================================================-% 

  %+===================================================================+%
  \begin{frame}{Mechanisms of DMI \(\rhd\) Basis-set expansion} 
    \scriptsize
    \begin{block}{}
      \begin{equation}\begin{aligned}
        \label{eq::H}
        \widehat{H} = \widehat{H}_0^{\text{all}} + \widehat{T}^{\text{all}} &= \sum_{\bm{R}}\sum_n{}\epsilon_n(\bm{R})\left[\widehat{\alpha}_{n\uparrow}^\dagger(\bm{R})\widehat{\alpha}_{n\uparrow}(\bm{R}) + \widehat{\alpha}_{n\downarrow}^\dagger(\bm{R})\widehat{\alpha}_{n\downarrow}(\bm{R})\right] \\
        &+\sum_{\bm{R}\ne{}\bm{R}'}\sum_{n,n'}\left\{b_{n'n}(\bm{R}'-\bm{R})\left[\widehat{\alpha}_{n'\uparrow}^{\dagger}(\bm{R}')\widehat{\alpha}_{n\uparrow}(\bm{R}) + \widehat{\alpha}_{n'\downarrow}^{\dagger}(\bm{R}')\widehat{\alpha}_{n\downarrow}(\bm{R})\right]\right.\\
        &\hspace*{5.5em} + C_{n'n}^{z}(\bm{R}'-\bm{R})\left[\widehat{\alpha}_{n'\uparrow}^{\dagger}(\bm{R}')\widehat{\alpha}_{n\uparrow}(\bm{R}) - \widehat{\alpha}_{n'\downarrow}^{\dagger}(\bm{R}')\widehat{\alpha}_{n\downarrow}(\bm{R})\right]\\
        &\hspace*{5.5em} +C_{n'n}^{-}(\bm{R}'-\bm{R})\widehat{\alpha}_{n'\uparrow}^{\dagger}(\bm{R}')\widehat{\alpha}_{n\downarrow}(\bm{R})\\
        &\left.\hspace*{5.5em} +C_{n'n}^{+}(\bm{R}'-\bm{R})\widehat{\alpha}_{n'\downarrow}^{\dagger}(\bm{R}')\widehat{\alpha}_{n\uparrow}(\bm{R})\right\}
      \end{aligned}\end{equation} 
    \end{block}
    \begin{subequations}\begin{align}
      b_{n'n}(\bm{R}'-\bm{R}) + C_{n'n}^{z}(\bm{R}'-\bm{R}) &= \int\omega_{n'\uparrow}^*(\bm{r}-\bm{R}')H_1\omega_{n\uparrow}(\bm{r}-\bm{R})\mathrm{d}\bm{r}\\ 
      b_{n'n}(\bm{R}'-\bm{R}) - C_{n'n}^{z}(\bm{R}'-\bm{R}) &= \int\omega_{n'\downarrow}^*(\bm{r}-\bm{R}')H_1\omega_{n\downarrow}(\bm{r}-\bm{R})\mathrm{d}\bm{r}\\
      \gray{C_{n'n}^{x}(\bm{R}'-\bm{R}) - iC_{n'n}^{y}(\bm{R}'-\bm{R}) = \;} C_{n'n}^{-}(\bm{R}'-\bm{R}) &= \int\omega_{n'\uparrow}^*(\bm{r}-\bm{R}')H_1\omega_{n\downarrow}(\bm{r}-\bm{R})\mathrm{d}\bm{r}\\
      \gray{C_{n'n}^{x}(\bm{R}'-\bm{R}) + iC_{n'n}^{y}(\bm{R}'-\bm{R}) = \;} C_{n'n}^{+}(\bm{R}'-\bm{R}) &= \int\omega_{n'\downarrow}^*(\bm{r}-\bm{R}')H_1\omega_{n\uparrow}(\bm{r}-\bm{R})\mathrm{d}\bm{r}
    \end{align}\end{subequations}
  \end{frame}
  %-===================================================================-% 

  %+===================================================================+%
  \begin{frame}{Mechanisms of DMI \(\rhd\) Perturbation}
    Basically, there are three techniques to achieve our aims, that perturbed the low energy subspace with the high energy excitation space.
    \begin{itemize}
      \item Downfolding\footnote{\tiny\href{https://github.com/kYangLi/PublicPresentation/blob/master/Intro2MEM/Intro._to_MEM.pdf}{Introduction to the Magnetic Exchange Mechanisms, Yang Li, 2020}}
      \item Lowdin Perturbation\footnote{\tiny\url{https://link.springer.com/content/pdf/bbm\%3A978-1-4615-5673-2\%2F1.pdf}}
      \item Green's Function
    \end{itemize}
    \begin{block}{}
      All of those three methods gives the same result:
      \begin{equation}
        \widehat{H}_{\text{eff}} = \widehat{H}_0 - \frac{\widehat{T}^\dagger\widehat{T}}{U} = \widehat{H}_0 + \widehat{H}_{\text{M}}
      \end{equation}
      Where \(U\) is the Hubbard \(U\) refer to the energy cost when move two electrons to the same site. And \(\widehat{T}\) only content the hoping terms contribute to this process.
    \end{block}
  \end{frame}
  %-===================================================================-% 

  %+===================================================================+%
  \begin{frame}{Mechanisms of DMI \(\rhd\) Simplification}
    \begin{block}{}
      \begin{equation*}
        \widehat{H}_{\text{M}} = -\frac{1}{U}\widehat{T}^\dagger\widehat{T}
      \end{equation*}
      \begin{equation}
        \begin{aligned}
          \widehat{T} &= b_{n'n}(\bm{R}'-\bm{R})\left[\widehat{\alpha}_{n'\uparrow}^{\dagger}(\bm{R}')\widehat{\alpha}_{n\uparrow}(\bm{R}) + \widehat{\alpha}_{n'\downarrow}^{\dagger}(\bm{R}')\widehat{\alpha}_{n\downarrow}(\bm{R})\right]\\
          &+ C_{n'n}^{z}(\bm{R}'-\bm{R})\left[\widehat{\alpha}_{n'\uparrow}^{\dagger}(\bm{R}')\widehat{\alpha}_{n\uparrow}(\bm{R}) - \widehat{\alpha}_{n'\downarrow}^{\dagger}(\bm{R}')\widehat{\alpha}_{n\downarrow}(\bm{R})\right]\\
          &+ C_{n'n}^{-}(\bm{R}'-\bm{R})\widehat{\alpha}_{n'\uparrow}^{\dagger}(\bm{R}')\widehat{\alpha}_{n\downarrow}(\bm{R})\\
          &+ C_{n'n}^{+}(\bm{R}'-\bm{R})\widehat{\alpha}_{n'\downarrow}^{\dagger}(\bm{R}')\widehat{\alpha}_{n\uparrow}(\bm{R}) + h.c.
        \end{aligned}
      \end{equation}
    \end{block}
    The relation between fermi annihilation/creation
    operators and spin operators is, 
    \begin{subequations}
      \begin{align}
        \widehat{S}_{z,n}(\bm{R}) &= \dfrac{1}{2}\left[\widehat{\alpha}_{n\uparrow}^\dagger(\bm{R})\widehat{\alpha}_{n\uparrow}(\bm{R}) - \widehat{\alpha}_{n\downarrow}^\dagger(\bm{R})\widehat{\alpha}_{n\downarrow}(\bm{R})\right]\\
        \widehat{S}_{+,n}(\bm{R}) &= \widehat{\alpha}_{n\uparrow}^\dagger(\bm{R})\widehat{\alpha}_{n\downarrow}(\bm{R})\\
        \widehat{S}_{-,n}(\bm{R}) &= \widehat{\alpha}_{n\downarrow}^\dagger(\bm{R})\widehat{\alpha}_{n\uparrow}(\bm{R})
      \end{align}
    \end{subequations}
  \end{frame}
  %-===================================================================-% 

  %+===================================================================+%
  \begin{frame}{Mechanisms of DMI \(\rhd\) Results}
    \small
    \begin{equation}\begin{aligned}
      \widehat{H}_{M} &= J_{\bm{R},\bm{R}'}\;\widehat{\bm{S}}(\bm{R})\cdot\widehat{\bm{S}}(\bm{R}') \\
      &+ \purple{\bm{D}_{\bm{R},\bm{R}'}\cdot\left[\widehat{\bm{S}}(\bm{R})\times\widehat{\bm{S}}(\bm{R}')\right]} \\
      &+ \widehat{\bm{S}}(\bm{R})\cdot\overleftrightarrow\Gamma_{\bm{\bm{R},\bm{R}'}}\cdot\widehat{\bm{S}}(\bm{R}')
    \end{aligned}\end{equation}
    \scriptsize
    Where, 
    \begin{subequations}\begin{align}
      J_{\bm{R},\bm{R}'} &= 2|b_{nn'}(\bm{R}-\bm{R}')|^2/U\\
      \purple{\bm{D}_{\bm{R},\bm{R}'}} &\hspace*{0.3em}\purple{= (4i/U)\left[b_{nn'}(\bm{R}-\bm{R}')\bm{C}_{n'n}(\bm{R}'-\bm{R}) - b_{n'n}(\bm{R}'-\bm{R})\bm{C}_{nn'}(\bm{R}-\bm{R}')\right]}\\
      \overleftrightarrow\Gamma_{\bm{\bm{R},\bm{R}'}} &= 4/U\left[\bm{C}_{n'n}(\bm{R}'-\bm{R})\otimes\bm{C}_{nn'}(\bm{R}-\bm{R}') + \bm{C}_{nn'}(\bm{R}-\bm{R}')\otimes\bm{C}_{n'n}(\bm{R}'-\bm{R})\right. \nonumber \\
      &\hspace*{3.5em}- \left.\left(\bm{C}_{n'n}(\bm{R}'-\bm{R})\cdot\bm{C}_{nn'}(\bm{R}-\bm{R}')\right)\mathbbm{1}\right]
    \end{align}\end{subequations}
    And,
    \begin{equation}
      \begin{aligned}
        b_{nn'}(\bm{R}-\bm{R}') &= b_{n'n}^*(\bm{R}'-\bm{R})\\
        \bm{C}_{nn'}(\bm{R}-\bm{R}') &= \bm{C}_{n'n}^*(\bm{R}'-\bm{R})\\
        \bm{C} &= (C_x, C_y, C_z)
      \end{aligned}
    \end{equation}
  \end{frame}
  %-===================================================================-%

  %+===================================================================+%
  \begin{frame}{Mechanisms of DMI \(\rhd\) Symmetry protection}
    \begin{equation}
      \widehat{H}_{\text{DM}} = \bm{D}\cdot(\widehat{\bm{S}}_1\times\widehat{\bm{S}}_2)
    \end{equation}
    Suppose the two ions contribute to the electron transfer is located at point \(A\) and \(B\), the point bisecting \(AB\) is denoted by \(C\). 
    \begin{block}{}
      \begin{itemize}
        \item When a center of inversion if located at \(C\), \purple{\(\bm{D} = 0\)}
        \item When a mirror plane perpendicular to \(AB\) passes through \(C\), \purple{\(\bm{D}\parallel{}\text{mirror plane}\)}
        \item When a mirror plane including \(AB\), \purple{\(\bm{D}\perp{}\text{mirror plane}\)}
        \item When a \(C_2\) axis perpendicular to \(AB\) passes \(C\), \purple{\(\bm{D}\perp{}C_2\;\text{axis}\)}
        \item When there is a \(C_n\) (\(n\geq{}2\)) alone \(AB\), \purple{\(\bm{D}\parallel{}AB\)}
        
      \end{itemize} 
    \end{block}
  \end{frame}
  %-===================================================================-%
\end{document} 